%%%%%%%%%%%%%%%%%%%%%%%%%%%%%%%%%%%%%%%%%
% Masters/Doctoral Thesis 
% LaTeX Template
% Version 2.5 (27/8/17)
%
% This template was downloaded from:
% http://www.LaTeXTemplates.com
%
% Version 2.x major modifications by:
% Vel (vel@latextemplates.com)
%
% This template is based on a template by:
% Steve Gunn (http://users.ecs.soton.ac.uk/srg/softwaretools/document/templates/)
% Sunil Patel (http://www.sunilpatel.co.uk/thesis-template/)
%
% Template license:
% CC BY-NC-SA 3.0 (http://creativecommons.org/licenses/by-nc-sa/3.0/)
%
%%%%%%%%%%%%%%%%%%%%%%%%%%%%%%%%%%%%%%%%%

%----------------------------------------------------------------------------------------
%	PACKAGES AND OTHER DOCUMENT CONFIGURATIONS
%----------------------------------------------------------------------------------------

\documentclass[
11pt, % The default document font size, options: 10pt, 11pt, 12pt
%oneside, % Two side (alternating margins) for binding by default, uncomment to switch to one side
english, % ngerman for German
singlespacing, % Single line spacing, alternatives: onehalfspacing or doublespacing
%draft, % Uncomment to enable draft mode (no pictures, no links, overfull hboxes indicated)
%nolistspacing, % If the document is onehalfspacing or doublespacing, uncomment this to set spacing in lists to single
%liststotoc, % Uncomment to add the list of figures/tables/etc to the table of contents
%toctotoc, % Uncomment to add the main table of contents to the table of contents
parskip, % Uncomment to add space between paragraphs
%nohyperref, % Uncomment to not load the hyperref package
headsepline, % Uncomment to get a line under the header
%chapterinoneline, % Uncomment to place the chapter title next to the number on one line
%consistentlayout, % Uncomment to change the layout of the declaration, abstract and acknowledgements pages to match the default layout
]{master-thesis} % The class file specifying the document structure
\restoreparindent
\usepackage[utf8]{inputenc} % Required for inputting international characters
\usepackage[T1]{fontenc} % Output font encoding for international characters
\usepackage{amsmath}
\usepackage{hyperref}
%\usepackage{mathpazo} % Use the Palatino font by default
%\usepackage{xfrac,unicode-math}
\usepackage[eng,exjobb]{KTHEEtitlepage}
\usepackage{amssymb}
\usepackage{bm}
\usepackage{cancel}
\usepackage{booktabs, lscape}

\usepackage[backend=bibtex,style=authoryear,natbib=true]{biblatex} % Use the bibtex backend with the authoryear citation style (which resembles APA)
%\setlength{\intextsep}{10pt plus 1.0pt minus 2.0pt}
\addbibresource{biblio.bib} % The filename of the bibliography
\usepackage{float}
\usepackage[autostyle=true]{csquotes} % Required to generate language-dependent quotes in the bibliography
\usepackage{subcaption}
\newcommand{\overbar}[1]{\mkern 1.5mu\overline{\mkern-1.5mu#1\mkern-1.5mu}\mkern 1.5mu}
%\setmathfont{Latin Modern Math}[version=lm]
\newcommand{\ra}[1]{\renewcommand{\arraystretch}{#1}}

%----------------------------------------------------------------------------------------
%	MARGIN SETTINGS
%----------------------------------------------------------------------------------------

\geometry{
	paper=a4paper, % Change to letterpaper for US letter
	inner=2.5cm, % Inner margin
	outer=3.5cm, % Outer margin
	bindingoffset=.5cm, % Binding offset
	top=1.5cm, % Top margin
	bottom=1.5cm, % Bottom margin
	%showframe, % Uncomment to show how the type block is set on the page
}

%----------------------------------------------------------------------------------------
%	THESIS INFORMATION
%----------------------------------------------------------------------------------------

\thesistitle{Energy-based Multi-Modal Attention} % Your thesis title, this is used in the title and abstract, print it elsewhere with \ttitle
\supervisor{Dr. Raphaël \textsc{Marée}} % Your supervisor's name, this is used in the title page, print it elsewhere with \supname
\examiner{} % Your examiner's name, this is not currently used anywhere in the template, print it elsewhere with \examname
\degree{Master of Computer Science and Engineering} % Your degree name, this is used in the title page and abstract, print it elsewhere with \degreename
\author{Aurélien \textsc{Werenne}} % Your name, this is used in the title page and abstract, print it elsewhere with \authorname
\addresses{} % Your address, this is not currently used anywhere in the template, print it elsewhere with \addressname

\subject{Artificial Intelligence} % Your subject area, this is not currently used anywhere in the template, print it elsewhere with \subjectname
\keywords{Multimodal, Deep Learning, Attention, Robustness} % Keywords for your thesis, this is not currently used anywhere in the template, print it elsewhere with \keywordnames
\university{University of Liège} % Your university's name and URL, this is used in the title page and abstract, print it elsewhere with \univname
\faculty{Faculty of Applied Sciences} % Your faculty's name and URL, this is used in the title page and abstract, print it elsewhere with \facname
\department{Montefiore Institute} % Your department's name and URL, this is used in the title page and abstract, print it elsewhere with \deptname
\group{\href{http://www.montefiore.ulg.ac.be/systmod/}{Systems and Modelling}} % Your research group's name and URL, this is used in the title page, print it elsewhere with \groupname

\AtBeginDocument{
\hypersetup{pdftitle=\ttitle} % Set the PDF's title to your title
\hypersetup{pdfauthor=\authorname} % Set the PDF's author to your name
\hypersetup{pdfkeywords=\keywordnames} % Set the PDF's keywords to your keywords
}

\begin{document}

\frontmatter % Use roman page numbering style (i, ii, iii, iv...) for the pre-content pages

\pagestyle{plain} % Default to the plain heading style until the thesis style is called for the body content

%----------------------------------------------------------------------------------------
%	TITLE PAGE
%----------------------------------------------------------------------------------------
\ititle{Energy-based Multi-Modal Attention}
\idate{August 2019}
\color{white}\iauthor{Aurelien Werenne}
\makeititle
\color{black}
\begin{titlepage}
\begin{center}

%\vspace*{.06\textheight}
%{\scshape\LARGE \univname\par}\vspace{1.5cm} % University name
%\textsc{\Large Master Thesis}\\[0.5cm] % Thesis type

\vspace*{.06\textheight}
\begin{center}
\includegraphics[scale=0.6]{figures/logo-uliege}\\[1.6cm] 
\end{center}

\HRule \\[0.4cm] % Horizontal line
{\huge \bfseries \ttitle\par}\vspace{0.4cm} % Thesis title
\HRule \\[1.5cm] % Horizontal line
 
\begin{minipage}[t]{0.4\textwidth}
\begin{flushleft} \large
\emph{Author:}\\
\authorname 
\end{flushleft}
\end{minipage}
\begin{minipage}[t]{0.4\textwidth}
\begin{flushright} \large
\emph{Supervisor:} \\
\supname
\end{flushright}
\end{minipage}\\[2.4cm]

\large \textit{A thesis submitted in partial fulfillment of the requirements\\ for the degree of \degreename}\\[1cm] % University requirement text
\deptname\\
\facname\\
\univname\\
Liège, Belgium\\[4cm] 
 

%\begin{center}
%\includegraphics[scale=0.6]{figures/logo-uliege}\\[1.6cm] 
%\end{center}
{\large Academic Year 2018 - 2019}
 
\vfill
\end{center}
\end{titlepage}

%----------------------------------------------------------------------------------------
%	QUOTATION PAGE
%----------------------------------------------------------------------------------------

\vspace*{0.2\textheight}

\noindent\enquote{\itshape Sometimes it seems as though each new step towards Artificial Intelligence, rather than producing something which everyone agrees is real intelligence, merely reveals what real intelligence is not.}\bigbreak
\hfill Douglas Hofstadter
%\ldots

%----------------------------------------------------------------------------------------
%	ABSTRACT PAGE
%----------------------------------------------------------------------------------------

\begin{abstract}
\addchaptertocentry{\abstractname} % Add the abstract to the table of contents
A multi-modal neural network exploits information from different channels and in different terms (e.g., images, text, sounds, sensor measures) in the hope that the information carried by each mode is complementary, in order to improve the predictions of the neural network. Nevertheless, in realistic situations, varying levels of perturbations can occur on the data of the modes, which may decrease the quality of the inference process. An additional difficulty is that these perturbations vary between the modes and on a per-sample basis. This work presents a solution to this problem. The three main contributions are described below.

First, a novel attention module is designed, analysed and implemented. This attention module is constructed to help multi-modal networks handle modes with perturbations.

Secondly, two new regularizers are developed to generalize the robustness to more intensive failing modes (relative to the training set).

Lastly, a unified multi-modal attention module is presented, combining the main types of attention mechanisms in the deep learning literature with our module. We suggest that the unified module could be coupled with a prediction model to enable the latter face unexpected situations, and extract the most relevant information from the input data.

\end{abstract}

%----------------------------------------------------------------------------------------
%	ACKNOWLEDGEMENTS
%----------------------------------------------------------------------------------------

\begin{acknowledgements}
\addchaptertocentry{\acknowledgementname} % Add the acknowledgements to the table of contents
\vspace*{5mm}
I would like to thank everybody who kept me busy this year. In particular, my thesis advisor Dr. Raphaël Marée for always encouraging my research, and Romain Mormont who gave me valuable feedback on the writing of this Master thesis. I would also like to thank the jury for reading the text.

Moreover, I would like to acknowledge the work of all the Professors at the University of Liège who helped me become an engineer.

I am also very grateful to my good friends Mathias Berger and Lucas Fuentes. They took time of their busy schedules to provide me with very valuable comments and suggestions, which have drastically improved the quality of this thesis. Thank you. 

Finally, I must express my profound gratitude to my family, especially my grand-parents for providing me with unfailing support and continuous encouragement throughout the process of researching and writing this thesis. This accomplishment would not have been possible without them.\\[1cm]

\begin{flushright}
\textit{Aurélien Werenne}\\
Liège, Belgium 2018-2019
\end{flushright}

\end{acknowledgements}

%----------------------------------------------------------------------------------------
%	LIST OF CONTENTS/FIGURES/TABLES PAGES
%----------------------------------------------------------------------------------------

\tableofcontents % Prints the main table of contents

\listoffigures % Prints the list of figures

%----------------------------------------------------------------------------------------
%	SYMBOLS
%----------------------------------------------------------------------------------------
%Abbreviations and Notation
\begin{symbols}{ll}

$\triangleq$ & Is defined as \\
$N$ & Number of samples \\
$M$ & Number of modes\\
$k_B$ & Boltzmann constant \\
$M_{\bigodot}$ & Solar mass \\
$\mathrm{e}$ & Euler's number, base of the natural logarithm ($2.71828$) \\
$\mathcal{L}$ & Loss function \\
$\bm{\theta}$ & Set of parameters of the specified model \\
$\nabla_{\bm{\theta}}$ & Gradient with respect to $\bm{\theta}$ \\
$\lambda_c$ & Weight of capacity penalty \\
$\lambda_e$ & Weight of energy penalty \\
$\Omega$ & Energy regularizer \\
$\Psi_i$ & Potential energy of mode $i$ \\
$E_{\text{total}}$ & Total energy \\
$E_i$ & Modal energy of mode $i$ \\
$e_i$ & Self-energy of mode $i$ \\
$e_{ij}$ & Shared energy of mode $j$ on mode $i$ \\
$\alpha_i$ & Importance score of mode $i$ \\
$\beta_i$ & Attention score of mode $i$ \\
$\rho$ & Coldness in Boltzmann distribution \\
$T$ & Temperature in Boltzmann distribution\\


\addlinespace 
\addlinespace 

AE & \textbf{A}uto\textbf{e}conder\\
BP & \textbf{B}ack-\textbf{p}ropagation\\
CNN & \textbf{C}onvolutional \textbf{N}eural \textbf{N}etwork\\
DAE & \textbf{D}enoising \textbf{A}uto\textbf{e}conder\\
DL & \textbf{D}eep \textbf{L}earning\\
DM & \textbf{D}ispersion \textbf{M}easure\\
EMMA & \textbf{E}nergy-based \textbf{M}ulti-\textbf{M}odal \textbf{A}ttention\\
ISM & \textbf{I}nter\textbf{s}tellar \textbf{M}edium\\
IP & \textbf{I}ntegrated \textbf{P}rofile\\
LSTM & \textbf{L}ong \textbf{S}hort \textbf{T}erm \textbf{M}emory\\
MMDL & \textbf{M}ulti \textbf{M}odal \textbf{D}eep \textbf{L}earning\\
MMN & \textbf{M}ulti \textbf{M}odal \textbf{N}etwork\\
NLL & \textbf{N}egative \textbf{L}og-\textbf{L}ikelihood\\
RNN & \textbf{R}ecurrent \textbf{N}eural \textbf{N}etwork\\
SGD & \textbf{S}tochastic \textbf{G}radient \textbf{D}escent\\
SNR & \textbf{S}ignal-to-\textbf{n}oise \textbf{R}atio\\
WER & \textbf{W}ord \textbf{E}rror \textbf{R}ate\\

\end{symbols}

%----------------------------------------------------------------------------------------
%	THESIS CONTENT - CHAPTERS
%----------------------------------------------------------------------------------------

\mainmatter % Begin numeric (1,2,3...) page numbering

\pagestyle{thesis} % Return the page headers back to the "thesis" style

% Include the chapters of the thesis as separate files from the Chapters folder
% Uncomment the lines as you write the chapters

% Chapter Template

\chapter{Introduction} 
\label{chapter-introduction} 

%----------------------------------------------------------------------------------------
%	SECTION 
%----------------------------------------------------------------------------------------

\section{Motivation}

In recent years, tremendous progress has been made in the field of Artificial Intelligence (AI), especially in Deep Learning \citep{lecun-dl, deeplearning-overview}. Deep Learning has helped AI systems reach and sometimes surpass human-level perception, mainly in computer vision \citep{image-recognition} and natural language processing \citep{machine-translation}. This has given rise to amazing industrial applications such as autonomous driving, early cancer detection, enhanced machine translation, etc. In safety-critical contexts, a key concern of engineers is to make sure the trained models are error-free, which can be challenging if the input data does not hold sufficient information to reduce the uncertainty on the predictions to an admissible level.

One possible solution researchers have been exploring is to use multiple modalities\footnote{The term modality, also called mode, is generally understood to mean "the way in which something happened or is experienced" \citep{taxomany-multimodal}}, which has largely been inspired by the often multi-modal nature of information gathering processes in humans, i.e., we see objects, hear sound, feel the texture, smell odours, and taste flavours. Multi-modal deep learning (MMDL) essentially consists in exploiting information from different channels and in different forms in the hope that the information carried by each mode is (partially) complementary, in order to improve the predictive performance of deep learning models. For example, in \citep{lidar-camera} sensorial inputs from wide-angle cameras and LIDAR\footnote{Laser Detection and Ranging} sensors are combined for road detection. Cameras provide dense information over a long range under good illumination conditions and fair weather, whereas LIDARs are only marginally affected by the external lighting conditions but have a limited range. Thus, merging the complementary information of the two sensors improves the accuracy of the road detection process. Despite its improvements on the predictions, MMDL still suffers from a major drawback. Indeed, no systematic mechanisms exist to handle failing modes. In the present report, a mode is said to be failing if a) it has a high noise to signal ratio, b) the data is much different from the training data, c) the data is missing. Failing modes a) and b) generally degrade the quality of the predictions because they introduce perturbations in the neural network.

While a solution has not been found for neural networks, humans seem to handle these situations robustly on a daily basis. Analysing human strategies and translating them as much as possible into the framework of AI can provide interesting heuristics to solve this crucial issue. A famous example showing this human ability is called the cocktail-party effect \citep{cocktail-party}. It refers to the difficulty we sometimes have understanding speech in noisy social settings. As a subconscious response, we tend to look at the mouth of our interlocutor i.e. we shift some attention from the auditory to the visual senses. Similarly, our attention is shifted from vision to touch when we are in a room where the lights suddenly go out. These examples indicate that humans handle modes with perturbations (first example) or missing information (second example) by shifting their attention on to the other more relevant modes \citep{crossmodal}.

Inspired by this behaviour, this report presents a new approach to tackle failing modes. More precisely, a novel attention mechanism\footnote{"Attention mechanisms in deep learning aim to highlight specific regions of the input space" \citep{attentive-survey}}, dubbed \textit{Energy-based Multi-Modal Attention} (EMMA), is introduced. This mechanism determines how much attention to devote to each mode, so that the relevant information is kept while masking out the perturbations. Additionally, this work offers some insight into how other attention mechanisms in the deep learning literature resemble to the ones observed in humans.

\begin{figure}[!ht]
\centering
\includegraphics[scale=0.55]{figures/lidar-camera}
\caption[Lidar \& Camera view in self-driving cars]{Same environment, different modes (top: LIDAR view, bottom: camera view).}	
\label{fig:lidar-camera}
\end{figure}


%----------------------------------------------------------------------------------------
%	SECTION 
%----------------------------------------------------------------------------------------

\section{Proposed solution}\label{sec:proposed-solution}
In essence, the EMMA module is placed in front of and coupled with the deep learning model. The module multiplies each mode of the input sample by a weight so that the modes with the most useful features are amplified, while the modes that are unnecessary or contain too much perturbations are masked out. The amount of attention allocated to each mode is determined based on its importance, which is defined in terms of three intrinsically related properties, namely
\begin{itemize}
\item \textit{relevance}: the intrinsic informativeness of the mode for the predictive task at hand.
\item \textit{failure intensity}: the propensity of a mode to trigger undesirable activations in the neural network.\footnote{We suggest that a mode of the input sample that is significantly different from the training distribution may cause undesired activations in the neural network, thereby negatively affecting the predictions.}
\item \textit{coupling}: the interdependencies between the modes, which describe the extent to which the mode provide independent, complementary, redundant or conflicting information.
\end{itemize}
The module is designed in such a way that it is able to learn these three properties (and their interactions) for each mode. In other words, the module determines the attention to allocate to each mode on its general predictive power, the amount of perturbations it contains and the relationship it has with other modes. For example, this allows the module to mask out a specific failing mode provided that another mode can compensate for its failures. Let us stress that the determination of importance is done on a sample-per-sample basis.
\begin{figure}[!h]
\centering
\includegraphics[scale=0.45]{figures/introduction-three-modes-with-emma}
\caption[Multi-Modal model with/without EMMA]{A multi-modal model with three input modes, without EMMA (left), augmented with EMMA (right).}	
\label{fig:main-idea}
\end{figure}

\subsection*{Software Implementation}
All the implemented models and experiments are available at \href{https://github.com/Werenne/energy-based-multimodal-attention}{this}\footnote{\url{https://github.com/Werenne/energy-based-multimodal-attention}} repository, with a wiki explaining how to run the experiments; \href{https://pytorch.org/}{PyTorch}\footnote{\url{https://pytorch.org/}} \citep{paszke} was the main framework used for the Machine Learning part.

%----------------------------------------------------------------------------------------
%	SECTION 
%----------------------------------------------------------------------------------------

\section{Contributions}
The contributions of this Master thesis can be summarised as follows
\begin{description}
\item \textbf{Contribution 1: an attention module improving the robustness against failing modes.} In Chapter \ref{chapter-emma}, the design of a new attention mechanism based on energy models \citep{ebm-tutorial} is discussed, that can be added to any multi-modal model. 
\item \textbf{Contribution 2: a simple yet powerful regularizer applying to attention mechanisms.} A common attention function is modified, establishing a link to the concept of capacity of psychology \citep{attention-is-effort}, which pertains to the amount of attention allocated among inputs. Subsequently, a new regularizer is introduced to control the capacity, whose purpose is to help generalise against unexpected situation.
\item \textbf{Contribution 3: a unified model for multi-modal attention.} In Chapter \ref{chapter-literature-review}, a review of the literature on attention in humans helps us identify how to construct a more complete multi-modal attention module.
\end{description}

%----------------------------------------------------------------------------------------
%	SECTION 
%----------------------------------------------------------------------------------------

\section{Thesis Outline}
The remainder of this work is organised as follows.
\begin{description}
\item \textbf{Chapter 2} explains the background (i.e. deep learning and energy models) this work is based upon.
\item \textbf{Chapter 3} reviews the literature on attention in psychology and deep learning, and the similarities between them.
\item \textbf{Chapter 4} describes a method for the estimation of the failure intensity of a mode.
\item \textbf{Chapter 5} presents the ideas behind the architecture of the Energy-based Multi-Modal Attention module (Contribution 1 \& 2).
\item \textbf{Chapter 6} presents an evaluation and analysis of the module outlined in Chapter \ref{chapter-emma} (Contribution 1 \& 2).
\item \textbf{Chapter 7} proposes a unified multi-modal attention module (Contribution 3).
\item \textbf{Chapter 8} concludes this work and suggests possible directions for future research.
\end{description}
\chapter{Background} 
\label{chapter-background} 

%----------------------------------------------------------------------------------------
%	SECTION 
%----------------------------------------------------------------------------------------

\section{Machine Learning}
Machine Learning is a subfield of Artificial Intelligence (see Figure \ref{fig:venn-diagram}) concerned with the design of algorithms that allow machines (e.g. computers, robots, embedded systems) to learn. For a task \textbf{T}, a performance measure \textbf{P} and an amount of data \textbf{D}, the system is said to be learning if it improves its performance \textbf{P} at the task \textbf{T} by increasing \textbf{D} (gain experience). Moreover, there are three main types of learning paradigms, namely supervised, unsupervised and reinforcement learning. In supervised learning \citep{supervised}, the model learns on a labeled dataset, providing an answer that the algorithm can use to evaluate its accuracy on training data. An unsupervised model \citep{unsupervised}, on the contrary, extracts features and patterns from unlabelled data. Lastly, reinforcement learning \citep{reinforcement} is typically used to train agents in dynamic environments, where the agent is able to act upon the environment. Reinforcement learning is best explained by an analogy. The learning algorithm is like a dog trainer, which teaches the dog (agent) how to respond to specific signs, like a whistle for example. Whenever the dog responds correctly, the trainer gives a reward to the dog, reinforcing the correct behaviour of the dog. Based on these three paradigms, several families of algorithms were invented. Deep learning \citep{deeplearning-overview} is one of those families and is particularly powerful on perception tasks.
\begin{figure}[!ht]
\centering
\includegraphics[scale=0.45]{figures/venn}
\caption{Venn diagram of the Artificial Intelligence field}	
\label{fig:venn-diagram}
\end{figure}


%-----------------------------------
%	SUBSECTION 
%-----------------------------------
\section{Deep Learning}
Deep learning models, also called Deep Neural Networks, offer the significant advantage of being able to learn their own feature representation for the completion of a given task. A neural network is loosely inspired from our own brains, but can best be seen as a series of stacked non-linear parametric functions, enabling the network to learn multiple levels of representation with increasing abstraction. The parameters are tuned by optimizing a loss function with Stochastic Gradient Descent (SGD) or one of its many enhancements \citep{optim-algos}. Let $\bm{\theta}$ be the set of parameters, $\mathcal{L}$ the loss function, $y$ the groundtruth (labels) and $\hat{y}$ the predictions. First, the SGD algorithm estimates the gradient of the cost function on a randomly sampled batch of size $N$ as
\begin{equation}
\mathbf{g} = \frac{1}{N}\nabla_{\bm{\theta}}\sum_{i=1}^N\mathcal{L}(\hat{y}^{(i)},y^{(i)})
\end{equation}
where the computation of the gradient itself is done using back-propagation (BP) \citep{backprop}. The SGD algorithm then follows the estimated gradient downhill, $\bm{\theta} \leftarrow \bm{\theta} - \epsilon\mathbf{g}$ where $\epsilon$ is the learning rate, in the hope of minimizing the loss.

Optimizing the parameters to represent all valid inputs of a task, where the data is often very high-dimensional (e.g., images, sounds, text), may seem hopeless. However, neural networks surmount this obstacle by assuming that these high-dimensional data are lying along low-dimensional manifolds\footnote{A manifold designates a connected set of points that can be approximated well by considering only a small numbers of degrees of freedom} \citep{goodfellow-book}. An intuitive observation in favour of this claim is that uniform noise essentially never resembles structured inputs from these tasks. More rigorous experiments supporting the manifold hypothesis are \citep{manifold-1, manifold-2, manifold-3}.


\subsection*{Multi-Modal Deep Learning}\label{sec:mmdl}
As a reminder, a modality refers to "the way in which something happened or is experienced" \citep{taxomany-multimodal}. Multi-Modal Deep Learning (MMDL) is simply the research area of neural networks using input samples consisting of multiple modes. Baltrušaitis et al. identified five non-exclusive use-cases of MMDL,
\begin{itemize}
\item \textit{Representation}: learning how to represent and summarize multi-modal data in a way that exploits the complementarity and redundancy
\item \textit{Translation}: learning how to map data from one modality to another (e.g., image captioning)
\item \textit{Alignment}: learning to identify the direct relationships between elements from two or more different modalities (e.g. alignment of sound and video)
\item \textit{Fusion}: learning to join information from two or more modalities to perform predictions 
\item \textit{Co-learning}: learning to transfer knowledge between modalities and their respective predictive models (e.g., zero shot learning)
\end{itemize}
The EMMA module is applied to multi-modal networks performing fusion. Furthermore, networks doing fusion can combine their modalities in three different ways: by early-fusion, late-fusion and an hybrid of the first two. Early-fusion architectures have uni-modal encoders extracting the features of each mode, the obtained features are then concatenated altogether and fed into a common decoder making the predictions (see Figure \ref{fig:early-fusion}). In contrast, late-fusion has uni-modal predictors for each mode, followed by a decoder weighting the uni-modal predictions to compute the final prediction. 
\begin{figure}[!h]
\centering
\begin{subfigure}{.45\textwidth}
\vspace*{15.5mm}
  \centering
  \includegraphics[width=.55\linewidth]{figures/early-fusion}
  \vspace*{3mm}
  \caption{Early-fusion}
  \label{fig:early-fusion}
\end{subfigure}%
\begin{subfigure}{.45\textwidth}
  \centering
  \includegraphics[width=.55\linewidth]{figures/late-fusion}
  \vspace*{2mm}
  \caption{Late-fusion}
  \label{fig:late-fusion}
\end{subfigure}
\caption[Early and late fusion]{Fusion of images and sounds for a classification task with a Convolutional Neural Network (CNN) \citep{image-recognition}, Recurrent Neural Network (RNN) \citep{machine-translation}}
\label{fig:fusion}
\end{figure}


%----------------------------------------------------------------------------------------
%	SECTION 
%----------------------------------------------------------------------------------------
\section{Physics meets Deep Learning}\label{sec:ebm}
Modelling complex probability distributions by parametric functions such as deep learning models is a difficult task, because all the probabilities must be positive and sum up to one. At its origins, many researchers in deep learning had an academic background in physics, from which they regularly found inspiration to solve problems. An example of this is the distribution of kinetic energies among molecules of gas, called the Boltzmann distribution, and given by
\begin{equation}
p(E_i) = \frac{1}{Z}e^{-E_i/k_B T} \quad \text{with the partition function} \quad Z = \int e^{-E_j/k_B T}
\label{eq:boltzmann-distrib}
\end{equation}
where $E_i$ is the kinetic energy of molecule $i$, $k_B$ the Boltzmann constant and $T$ the temperature of the environment. The first thing to notice is that all the probabilities are positive and sum up to one for any set of combinations of energies $E_i$. Another observation to make is that high values of energies are unlikely ($E_i \propto -\log p(E_i)$), unless the temperature is sufficiently high enough. To sum it up, the Boltzmann distribution can be used to normalize any function to a distribution, where the temperature $T$ is a parameter influencing the entropy of the distribution. The Boltzmann distribution has two major applications in deep learning. First, it corresponds to the soft-max activation function \citep{softmax}, employed commonly for the purpose of outputting probabilities in multi-category classification tasks. Secondly, it was also used by deep learning researchers to construct energy-based models \citep{ebm-tutorial}. These types of neural networks optimize an energy function to be low on the data manifold and high everywhere else (see Figure \ref{fig:ebm-intervals}), which is the mapped to probabilities via the Boltzmann distribution. A few examples of efficient energy-based models are Generative Adversarial Networks (GAN) \citep{gan}, Variational Autoencoders (VAE) \citep{kingma-vae} and Denoising Autoencoders (DAE) \citep{dae-vincent}. The latter will be used in this work to measure the outlyingness of the data (see Chapter \ref{chapter-energy-estimation}).
\begin{figure}[!ht]
\centering
\includegraphics[scale=1.15]{figures/ebm-intervals}
\caption[Energy surface evolution]{The shape of the energy surface at four intervals. Along the x-axis is the variable X and along the y-axis is the variable Y . The shape of the surface at (a) the start of the training, (b) after 15 epochs over the training set, (c) after 25 epochs, and (d) after 34 epochs. The energy surface has attained the desired shape: the energies around the training samples are low and energies at all other points are high. \textit{Image and caption from} \,\citep{ebm-tutorial}}
\label{fig:ebm-intervals}
\end{figure}





 
\chapter{Literature Review}\label{chapter-literature-review} 

The purpose of this chapter is to review the state-of-the-art literature of multi-modal attention. The first section describes attention in humans from both a psychological and a neurological point of view. We argue this will give the reader more intuition about attention in deep learning. The second part moves on to the different attention mechanisms in deep learning, in particular self-attention and crossmodal attention. 


%-----------------------------------
%	SECTION 
%-----------------------------------
\section{Attention in Humans}
The most profound effect of attention is its capacity to bring the attended stimuli into the forefront of our conscious experience while unattended stimuli fades into the background, increasing the processing efficiency at every stage of perception \citep{watzl}. A widely held assumption in the psychology literature is that the most fundamental function of attention is selection. At the level of single neurons, neuroscientists typically thought of attention in terms of selection between stimuli competing for the same neural receptive field \citep{neuro-level}. Daniel Kahneman, an authorithy in psychology and economy, investigated the way in which humans perform multi-tasking (i.e., solve a multi-modal problem). Kahneman claimed that attention was more than selection, that it could be viewed as a limited resource being shared among the different modes, but he could not generalize his findings to the intra-modal level\footnote{Intra-modal attention manifests itself only in a subset of the mode, whereas inter-modal attention is between modes.}. Moreover, the selection theory has been vigorously challenged in recent years by the amplification theory, where attention is an additional activity that interacts with built-in perceptual mechanisms by amplifying some of the input signals \citep{amplification}. Furthermore, the absolute intensity of amplification is not important, in contrary it is the relative intensity between the inputs that matters (\textit{the contrast effect}). Notice that the amplification theory generalizes the concept of capacity to the intra-modal level and neural level. Interestingly, we will see that the basic principles of attention mechanisms in deep learning has significant similarities with amplification.

Regarding multi-modal attention, three types can be distinguished: endogenous, exogenous and crossmodal attention \citep{crossmodal}. People orient their attention endogenously whenever they voluntarily choose to attend to something, such as when listening to a particular individual at a noisy cocktail party, or when concentrating on the texture of the object that they happen to be holding in their hands. By contrast, exogenous orienting occurs when a person’s attention is captured reflexively by the sudden onset of an unexpected event, such as when a mosquito suddenly lands on our arm. Lastly, crossmodal attention refers to the interaction of attention between two or more modes such as using visual clues (e.g. lip movements) to focus on the voice of a particular individual at a noisy cocktail party.


%-----------------------------------
%	SECTION 
%-----------------------------------
\section{Attention in Deep Learning}
Attention mechanisms in deep learning aim to highlight specific regions of the input space. The most common way to do this, is by multiplying the input by an attention mask, where the attention mask consist of normalized continous values between zero and one. Observe the similarity with the amplication theory described in the previous section. In self-attention \citep{bahdanau}, the attention mask is computed from the same mode on which it is applied. Conversely, for crossmodal attention mechanisms \citep{crossmodal-object-detection}, the attention mask is computed from multiple modes. 

Self-attention was first introduced in natural language processing (NLP) for machine translation tasks by \citep{bahdanau}. It helped the translation task by enabling the model to automatically search for parts of a source sentence that are relevant to predicting the next target work. With this approach, Bahdanau et al. achieved a translation performance comparable to the existing state-of-the-art phrase-based system on the task of English-French translation. Since then it has become a prominent tool in NLP but has also been used in a variety of other tasks such as image classification. \citep{self-capsule} uses self-attention to learn to suppress irrelevant regions in images and highlight salient features useful for the specific classification task. The authors in \citep{self-capsule} reduced the computation load and were able to compensate the absence of a deeper network by using the self-attention, without having a decreased classification performance. For a detailed review on this self-attention mechanisms, see \citep{attention-review}.

Turning now to crossmodal attention, \citep{looking-to-listen} presents an audio-visual model for isolating a single speech signal form a mixture of sounds such as other speakers and background noise (see Figure \ref{fig:looking-to-listen}). Crossmodal attention is used to focus on certain parts of the audio with respect to an image of the desired speaker. The authors showed superior results compared to state-of-the-art audio-only methods. Similar works \citep{cross-transformer, crossmodal-object-detection, crossmodal-video-caption} are using crossmodal attention and have attained impressive results. However, most research using crossmodal attention has tended to focus on obtaining better predictions rather than improving the robustness. A few exceptions are discussed below.
\begin{figure}[!ht]
\centering
\includegraphics[scale=0.85]{figures/look-to-listen}
\caption[Looking to Listen framework]{The authors of \citep{looking-to-listen} present a model for isolating and enhancing the speech of desired speakers in a video. Their model was trained using thousands of hours of video segments from our new dataset, AVSpeech. \textit{Image from} \citep{looking-to-listen}}
\label{fig:looking-to-listen}
\end{figure}

A work investigating how multimodal fusion can help against failing modes is \citep{afouras}. Their model fuses audio and video to obtain better speech-to-text. Interestingly, Afouras et al. use a combination of self-attention mechanisms followed by a crossmodal attention layer. The model was tested on thousands of natural sentences of British television. Furthermore, they added babble noise with 0dB signal-to-noise ratio to the audio streams, where the babble noise samples are synthesized by mixing the signals of 20 different audio samples from the dataset. The audio-visual model achieved a 13.7\% word error rate (WER) on the dataset without noise, and a 33.5\% WER on the dataset with noise whereas the audio-only model only achieved 64.7\% WER. Despite obtaining great results, a major weakness with this experiment, however, is that their test set is corrupted in the exact same manner as their training set. In our experiments\footnote{See Section \ref{sec:generalization}} we will show that evaluating test data with the same noise as on the training data can significantly overestimate the robustness of the model. Additionally, the attention module in \citep{afouras} is presumably not able to detect and handle unseen samples. To summarize, the model was not tested against realistic failing modes situations.

The work that is most relevant to our proposed method is the attentive context proposed in \citep{audiovisual-attention}, which also incorporates attention on the inter-modal level to explicitly filter perturbations out (see Figure \ref{fig:noise-tolerant}). The model is evaluated on a face-verification task, receiving a \textbf{v}oice sound and a \textbf{f}ace image. The attention mask $[\alpha_v,\alpha_f]$ is computed via a linear function, $f_{\text{att}} = \mathbf{W}^T[\mathbf{e}_v, \mathbf{e}_f] + \mathbf{b}$, on the embeddings $\mathbf{e}_v$ and $\mathbf{e}_f$. Several defects of the attention function $f_{\text{att}}$ can be observed: 
\begin{enumerate}
\item The function is unlikely to be expressive enough to capture complicated dependencies between the modes, and to recognize out-of-distribution\footnote{Relative to the training data} data.
\item A design constraint of this attention function is that all extracted embeddings must be of the same size, which may be a significant constraint when combining modes from low and high-dimensional data.
\item Similarly to \citep{afouras}, the test set is corrupted in the same manner as the training set.
\end{enumerate}

\begin{figure}[!ht]
\centering
\includegraphics[scale=0.85]{figures/noise-tolerant}
\caption[Noise-tolerant fusion model]{Neural network based fusion approaches. $\mathbf{e}_v$ : speaker embedding, $\mathbf{e}_f$ : face embedding. FC denotes a fully connected layer. \textit{Image from} \citep{audiovisual-attention}}
\label{fig:noise-tolerant}
\end{figure}




\chapter{Energy Approximators} 
\label{chapter-4} 

\begin{itemize}
\item As we will sse in next chapter, EMMA uses a measure of outlyingness, ideally true energy
\item However difficult to train, explore two simple alternative measures: ... those are approximations
\item Obtained training a DAE (see section below) explains the model and then next section explain where approximations come from.
\item Finally, we conclude with experiment to choose the best
\end{itemize}

%----------------------------------------------------------------------------------------
%	SECTION 
%----------------------------------------------------------------------------------------

\section{Autoencoder}

Principle AE. unsupervised training. MSE, x, r(x), recon, encoder, decoder. Why useful to do input. Let us distinguish two cases: under/overcomplete

%-----------------------------------
%	SUBSECTION 
%-----------------------------------
\subsection{Undercomplete}
\begin{itemize}
\item undercomplete, hidden representation, forces to keep only useful information, latent space...
\item explain diagrams. when we minimize MSE, we minimize norm of vector r-x
\end{itemize}

\begin{figure}[!h]
\centering
\begin{subfigure}{.5\textwidth}
  \centering
  \includegraphics[width=.75\linewidth]{figures/autoencoder-undercomplete}
  \caption{Autoencoder}
  \label{fig:ae}
\end{subfigure}%
\begin{subfigure}{.5\textwidth}
  \centering
  \includegraphics[width=.75\linewidth]{figures/reconstruction}
  \caption{Data space}
  \label{fig:ae-process}
\end{subfigure}
\caption{Reconstruction}
\label{fig:test}
\end{figure}

%-----------------------------------
%	SUBSECTION 
%-----------------------------------
\subsection{Overcomplete}

overcomplete to avoid copying - corruption becomes denoising. Can be used for denoising of signals. explain diagrams

\begin{figure}[!h]
\centering
\begin{subfigure}{.5\textwidth}
  \centering
  \includegraphics[width=.75\linewidth]{figures/autoencoder-overcomplete}
  \caption{Denoising Autoencoder}
  \label{fig:dae}
\end{subfigure}%
\begin{subfigure}{.5\textwidth}
  \centering
  \includegraphics[width=.75\linewidth]{figures/reconstruction-denoising}
  \caption{Data space}
  \label{fig:dae-process}
\end{subfigure}
\caption{Reconstruction with corruption}
\end{figure}

%----------------------------------------------------------------------------------------
%	SECTION 
%----------------------------------------------------------------------------------------

\section{Reconstruction norm \& Potential}

Small introduction. Explain article of Bengio intuitively and how it leads to score is reconstruction. 
$$ r(\tilde{\textbf{m}}_i) - \textbf{m}_i \propto \frac{\partial \log p(\textbf{m}_i)}{\partial \textbf{m}_i} $$ 
Explain how this relates to vector field. Show circle and vector field. \\
Then show idea of other paper intuitively. Potential. Conditions. Force.
$$\Psi_i = -\int (r(\tilde{\textbf{m}}_i) - \textbf{m}_i)d\textbf{m}_i$$
Develop not show, just mention it is proofed in paper and not difficult.
$$ \Psi_i = -\int f(\textbf{m}_i)d\textbf{m}_i - \frac{1}{2} \lVert \textbf{m}_i + \textbf{b}_r \rVert_2^2 + \text{const} $$
in this work sigmoid is only activation used but other can be easliy used (see paper).
$$ \Psi_i =  -\sum_k \log(1 + \text{exp}(W_{.k}^T \textbf{m}_i + b_k^h)) + \frac{1}{2} \lVert \textbf{m}_i - \textbf{b}_r \rVert_2^2 + \text{const}$$


%----------------------------------------------------------------------------------------
%	SECTION 
%----------------------------------------------------------------------------------------

\section{Experiment I}

We expect potential be better, because better grounded. 
Describe generation manifolds + parametric functions. Train AE with noise. experimental setup in appendix. Evaluate on grid. Two manifolds formulas. Mention manifold and their particular form were not chosen for a particular reason.
\begin{figure}[!h]
\centering
\begin{subfigure}{.5\textwidth}
  \centering
  \includegraphics[width=.4\linewidth]{figures/logo-uliege}
  \caption{Wave manifold}
  \label{fig:wave-manifold-only}
\end{subfigure}%
\begin{subfigure}{.5\textwidth}
  \centering
  \includegraphics[width=.4\linewidth]{figures/logo-uliege}
  \caption{Circle manifold}
  \label{fig:circle-manifold-only}
\end{subfigure}
\caption{Two manifolds}
\end{figure}

Show results+ observatiions (explain why there are sources, explain why less convergence in center of wave (because more accumulation)) + conclusion we will use potential (better results, more robuts, more theoretically grounded). Table of six figures.

%----------------------------------------------------------------------------------------
%	SECTION 
%----------------------------------------------------------------------------------------

\section{Limitations}

Explain easy but not easily extensible to images/sounds. Mention alternatives. Our goal is not to analyze best energy approximator bt to show that roughly any approx can be used to solve our problem.
\href{https://arxiv.org/pdf/1606.03439.pdf}{alternative} 
\chapter{Energy-based Multi-Modal Attention} 
\label{chapter-emma} 

In the Literature review (Chapter \ref{chapter-literature-review}) we concluded that previous work in multi-modal deep learning has been mostly focused on trying to increase the accuracy of the prediction. Few research has explicitly addressed the question of how using multiple modalities can improve the robustness. Inspired by how humans handle multiple senses robustly, I created a novel generic module that can easily be inserted into every trained multi-modal architecture. This chapter describes the ideas and architecture of the Energy-based Multi-Modal Attention module.

%----------------------------------------------------------------------------------------
%	SECTION 
%----------------------------------------------------------------------------------------

\section{Problem Statement}\label{sec:prob-statement}

We define the i.i.d. dataset $\mathcal{D}^{(N)}$ with $N$ samples  $(\mathbf{X},y)$. The input $\textbf{X}$ is composed of $M$ modes $\{\mathbf{x}_1, \ldots, \mathbf{x}_M\}$ of possibly different dimensions (e.g. images and sounds). The multi-modal network will be abbreviated as MMN. The model tries to make predictions $\hat{y}$ as close as possible to the groundtruth $y$ (see Figure \ref{fig:mnn}). The internal architecture of the MMN is often structured as a many-to-one encoder-decoder, where each encoder extracts features and the decoder is in charge of merging those together. Nonetheless, the EMMA module is not constrained to any specific MMN architecture.
\begin{figure}[!h]
\centering
\includegraphics[scale=0.5]{figures/mlp-without-emma}
\caption{High-level view of a Multi-Modal Network}	
\label{fig:mnn}
\end{figure}

Current research in MMDL is mostly motivated by how to make use of the information gain of adding an extra mode to make better predictions. In this work, we want to leverage this same information gain to improve the robustness. We start from the assumption that for a sample where one mode $i$ is outlying, it is likely to find at least another mode $j$ who is not outlying. This permits us to think that if we find a way in that case to shift the attention from mode $i$ to mode $j$, the performance would be better off. This is done by computing an importance score $\alpha_i$ for each mode $i$ between zero and one, giving us the relative importance. The importance scores $\alpha_i$ are a measure of how valuable each mode $i$ is taking into account the outlyingness of all the modes. From those, we determine the attention scores $\beta_i$, representing the quantity of information that can pass through. Each mode is then multiplied by its respective attention score (see Figure \ref{fig:mnn-with-emma}). We justify later on why we do not directly multiply be the importance score instead. TODO: explain how missing values problem is implicitly solved
\begin{figure}[!h]
\centering
\includegraphics[scale=0.5]{figures/mlp-with-emma}
\caption[High-level view of a Multi-Modal Network \& EMMA]{High-level view of a Multi-Modal Network with the EMMA module}	
\label{fig:mnn-with-emma}
\end{figure}

There are two ways of interpreting this solution. First, EMMA can be seen as a sort of gate filtering perturbations out. Indeed, outlying modes can provoke high activations in the MMN, disturbing the predictions. But by masking the outlying modes we diminish those activations, making it easier for the MMN to make good preditions. Another way to view it, is to understand that the MMN model easily extracts $\beta_i$ and $\mathbf{x}_i$ from the multiplication. The model then learns to make more robust predictions based on the extra inputs $\beta_i$.

In Section \ref{sec:general-framework}  we describe the different steps along with their main objectives. Hereafter, each section will correspond to a specific step and will detail how it works. Next, the training of the module is explained along with some regularizers. The chapter ends with an enumeration of the main research questions that needs to be addressed in the experiments. 

%----------------------------------------------------------------------------------------
%	SECTION 
%----------------------------------------------------------------------------------------

\section{General Framework}\label{sec:general-framework}

As we just saw, the model needs to compute how important each mode is. To do this let us start by introducing three intrinsically tied properties describing the importance of a mode $i$:
\begin{itemize}
\item \textit{relevance}: how much does mode $i$ help improve the the accuracy?
\item \textit{outlyingness}: is the current sample much different from the training set? How much will it import the predictions?
\item \textit{coupling}: does mode $j$ has a strong influence on mode $i$? If so in which way? Does mode $i$ and $j$ carry complementary or redundant information?
\end{itemize}
We define the modal energy $E_i$ for a mode $i$ as an embedding of the three properties. Modal energies are constructed as learnable parametric functions of potential energies,
\begin{equation}
E_i = f(\Psi_i)  + \sum_{k\neq i}^M g[f(\Psi_i), f(\Psi_k)]
\label{eq:general-framework}
\end{equation}
The function $f$ is able to capture the relationship between relevance and outlyingness. Because $f$ is optimized with a loss on the predictions and is also a function of the potential $\Psi_i$. The role of the function $g$ is to learn the optimal coupling between modes. Modal energies are then normalized to the importance scores. Which is in some kind equivalent of going from a measure of absolute importance to one of relative importance.

Attention is viewed in psychology as a selection process between senses or more generally, modes. Deep Learning research regarding attention is in majority based on this view. In contrast, the famous economist and psychologist Daniel Kahneman sees attention as a shared resource with a limited capacity being allocated between the modes \citep{attention-is-effort}. We mimic the latter by slightly modifying a common attention function. This is done with the intention of improving the interpretability of EMMA (see Section \ref{sec:capacity}). In this work, we use attention to decide how much information of a certain mode will pass.

\begin{figure}[!ht]
\centering
\includegraphics[scale=0.4]{figures/framework}
\caption[Summary of main steps in EMMA]{Summary of main steps in EMMA (step 2, 3 and 4 are detailed in the following sections, step 1 was explained in Chapter \ref{chapter-energy-estimation})}
\end{figure}

%----------------------------------------------------------------------------------------
%	SECTION 
%----------------------------------------------------------------------------------------

\section{From Potential to Modal energies (step 2)}
A problem overlooked so far is that we neglected the integration constant in the computation of the potential energy\footnote{Equation (\ref{eq:potential-prop})}. The potential can therefore take negative values. This is a problem because evaluating the gradient during the backpropagation involves taking a logarithm of $\Psi_i$\footnote{See Appendix \ref{chapter-misc}}, which is undefined for negative values. This problem is corrected by lowing the potential $\Psi_I$ to Euler's number $\mathrm{e}$ as
\begin{equation}
\Psi_i' = \max(\mathrm{e}, \Psi_i - \Psi_i^{(\text{min})} + \mathrm{e})
\end{equation}
With $\Psi_i^{(\text{min})}$ the lowest value of $\Psi_i$ in the training set. This correction avoids undefined values ($\Psi_i' \geq 0$) but also exploding gradient ($\Psi_i' \geq e$). The reason a max-operator is used is because lower energy values than $\Psi_i^{(\text{min})}$  can occur during inference.

The \textit{self-energy} of mode $i$ is defined as an energy capturing the outlyingness and the relevance of the mode. We write,
\begin{equation}
e_i = w_i\Psi_i' + b_i\,\,\, (=f(\Psi_i)), \qquad w_i, b_i \in \mathbb{R}^+
\end{equation}
The parameters $w_i$, $b_i$ are trained via a loss function on the predictions, thus adding the influence of the relevance. It also enables the model to face potentials on possibly very different scale, caused by the proportionality in Equation (\ref{eq:potential-prop}).

Furthermore, the \textit{shared energy} of mode $j$ on $i$ is constructed from the self-energies as follows
\begin{equation}
e_{ij} = w_{ij}e_i^{\gamma_{ij}}e_j^{1-\gamma_{ij}}\,\,\, (=g[f(\Psi_i), f(\Psi_k)]), \qquad w_{ij} \in \bigg[-\frac{1}{M-1}, +\frac{1}{M-1}\bigg],\,\, \gamma_{ij} \in [0,1]
\end{equation}
Adding the constraint $\gamma_{ij} = \gamma_{ji}$, the model can now discover the optimal coupling between the modes. It if learns a $\gamma_{ij}$ close to zero, mode $i$ and $j$ will influence each other much more than a $\gamma_{ij}$ near to one. The parameter $\gamma_{ij}$ learns the degree of coupling in the spectrum from strongly coupled ($\gamma_{ij}=0$) to independent ($\gamma_{ij} = 1$). The direction of coupling between mode $i$ and $j$ are learned by the weights $w_{ij}$ and $w_{ji}$. For a positive $w_{ij}$, an increase in self-energy $e_j$ causes an increase in $e_{ij}$. Whereas if $w_{ij}$ is negative, an increase in $e_j$ leads to a decrease in $e_{ij}$. The weights $w_{ij}$ are not imposed to be equal to $w_{ji}$, such that modes can influence each other asymmetrically. This assymetry is justified by the following example: a multi-modal problem with three modes A, B and C. Imagine the case where if mode A is missing, it is optimal that mode B takes over. But if B is missing it is optimal for C to take over. This can only be modelled with asymmetry.

Finally, the \textit{modal energy} of a mode is the sum of its self-energy and the shared energies with all the other modes:
\begin{equation}
E_i = e_i + \sum_{k\neq i}^M e_{ij}
\end{equation}
We can recognize Equation \ref{eq:general-framework}. This gives us also the \textit{total energy}, $E_{\text{total}} = \sum_i E_i$, offering an intuitive way to measure how uncertain the model is about its predictions.

%----------------------------------------------------------------------------------------
%	SECTION 
%----------------------------------------------------------------------------------------

\section{From Modal energies to Importance scores (step 3)}
The importance scores are computed from the modal energies via the Gibbs distribution:
\begin{equation}
\alpha_i = \frac{1}{Z}e^{-\rho E_i} \quad \text{with the partition function} \quad Z = \sum_{k=1}^M e^{-\rho E_k} 
\label{eq:gibbs-distrib}
\end{equation}
This guarantess the scores to be normalized and summing up to one. A mode $i$ will be said to be important if its score is close to one (low modal energy $E_i$). The hyperparameter $\rho$ represents the coldness, the inverse of the temperature. It controls the entropy of the importance scores distribution. At high temperature ($\rho \rightarrow 0$) the distribution becomes more uniform, and at low temperature ($\rho \rightarrow +\infty$) the importance scores corresponding to the lowest energy tends to 1, while the others approach 0 (see Figure \,\ref{fig:gibbs}). Careful tuning is thus necessary.

\begin{figure}[!h]
\centering
\begin{subfigure}{.5\textwidth}
  \centering
  \includegraphics[width=.95\linewidth]{figures/input-gibbs}
  \caption{Energies}
\end{subfigure}%
\begin{subfigure}{.5\textwidth}
  \centering
  \includegraphics[width=.95\linewidth]{figures/result-gibbs}
  \caption{Importance scores}
\end{subfigure}
\caption[Input-output of Gibbs distribution for two different temperatures]{Input-output of Gibbs distribution for two different temperatures, low temperature ($\rho = 0.1$) and high temperature ($\rho = 0.001$)}
\label{fig:gibbs}
\end{figure}


%----------------------------------------------------------------------------------------
%	SECTION 
%----------------------------------------------------------------------------------------

\section{From Importance to Attention scores (step 4)}\label{sec:capacity}
The attention scores are given by
\begin{equation}
\beta_i = \tanh(g_a\alpha_i - b_a) \quad \text{with} \quad g_a > 0,\,\,b_a\in [0,1]
\end{equation}
The hyperbolic tangent adds non-linearity while the gain $g_a$ and bias $b_a$ permits the model to control the threshold and capacity (see Figure \ref{fig:attention-function}). The latter two concepts are detailed below.

\subsection*{Energy threshold}
The module will let information in mode $i$ pass by only if $g_a\alpha_i - b_a > 0$
\begin{equation}
\begin{split}
&\Leftrightarrow\log(\alpha_i) > \log(b_a/g_a)\\
&\Leftrightarrow E_i \geq \frac{\log(g_a/b_a) - \log(Z)}{\rho} = E_{\text{threshold}}
\end{split}
\end{equation}
where $E_{\text{threshold}}$ represents the maximal amount of energy a mode $i$ is allowed to have in order to pass. As be seen the gain and bias control the threshold. Nevertheless, the value of the partition function $Z$ makes the threshold dynamic. The partition function will be higher if the modes are more outlying, thus diminishing the thresholds. In other words, EMMA adapts the selectiveness with respect to the quality of the data. TODO: This can be linked more arousal \href{http://www.scholarpedia.org/article/Crossmodal_attention}{link}. Notice that the influence of the temperature ($\rho^{-1}$) is non-trivial to analyse, because $Z$ also depends on $\rho$.

\subsection*{Capacity}
There is nothing new about using an hyperbolic tangent as an attention mechanism. The difference however resides in how the linear combination is restricted. A very common attention function is written as $\tanh(\mathbf{W}\mathbf{\alpha}+\mathbf{b})$ whereas we have $\tanh(g_a\mathbf{I}\mathbf{\alpha}-b_a\mathbf{u})$ with $\mathbf{u}$ the unit vector $(1 \ldots 1)^T$. We argue the latter mimics better human's crossmodal attention. The capacity in psychology is viewed as the amount of resource that can be allocated. This can be translated in our case as,
\begin{equation}
\text{capacity} \triangleq \int_0^1 \tanh(g_a\alpha + b_a)d\alpha 
\end{equation}
Define the auxiliary variable $u = g_a\alpha + b_a$. Now using
\begin{equation}
\frac{du}{d\alpha} = g_a \Leftrightarrow d\alpha = \frac{1}{g_a}du
\end{equation}
we can write 
\begin{equation}
\begin{split}
\text{capacity} &= \frac{1}{g_a} \int_0^1 \tanh(u)du  \\
&= \frac{1}{g_a}\log[\cosh(g_a\alpha + b_a)]\bigg\rvert_{\alpha = 0}^1 + \cancel{\text{constant}} \\
&= \frac{1}{g_a}\log\bigg[\frac{\cosh(g_a + b_a)}{\cosh(b_a)}\bigg]
\end{split}
\end{equation}
If the capacity is too low, there is no sufficient information passed to the MMN to make predictions and thus the accuracy drops. On the other hand, if the capacity is too high, too much perturbations will pass leading to bad performances. The module learns the optimal trade-off. Observe that the concept of capacity can also be applied to $\tanh(\mathbf{W}\mathbf{\alpha}+\mathbf{b})$. In that case, each mode would have a different capacity. This could make EMMA more expressive, but the importance scores would be less meaningful. Another advantage of having only one capacity is that it is easier to control. In the next section, we show a simple regularizer giving us some control on the capacity. The interest of doing this, is the idea that minimizing the capacity would allow to gain more robustness againts unseen situations at the cost of some accuracy. This regularizer is played along with in the experiments (see Chapter \ref{chapter-experiments}), giving us some interesting insight.

\begin{figure}[!h]
\centering
\includegraphics[scale=0.5]{figures/tanh-annotated}
\caption[Attention function]{Attention function, the max-operator generalizes the attention function to cases where $\alpha \in \mathbb{R}$}
\label{fig:attention-function}
\end{figure}
%----------------------------------------------------------------------------------------
%	SECTION 
%----------------------------------------------------------------------------------------

\section{Training \& Regularization}\label{sec:regul}
The end-to-end model (MMN \& EMMA) is trained in two phases (see Figure \ref{fig:training}). First, all the autoencoders are trained, one per mode. Once trained, the autoencoders are freezed. In the second phase, EMMA is inserted in front of the MMN and is trained end-to-end on both normal and corrupted data.

The motivation of EMMA is not only to improve predictions on the test-set. But also to see if it is able to handle new situations (robustness generalization) as expected. We also want EMMA to keep a certain level of interpretability. For those reasons we introduce two regularizers. The first one controls the capacity, where $\lambda_c$ can be positive/negative depending on if we want to maximize/minimize the capacity.
\begin{equation}
\tilde{\mathcal{L}} = \mathcal{L}(y,\hat{y}) + \lambda_c g_a - \lambda_e \Omega \quad \text{with} \quad \Omega = \sum_{k=1}^M \xi_k \log(\alpha_k) \quad \text{and} \quad \xi_k = \begin{cases}
      \xi_- = -1 & \text{if}\ \mathbf{x}_k\, \text{is corrupted} \\
      \xi_+ = +1 & \text{otherwise}
    \end{cases}
\label{eq:regularization}
\end{equation}
Additionally, regularizing the energy ($\lambda_e$) is done for interpretability purposes. If the parameters of modal energies are optimized only regarding the predictions ($\mathcal{L}$), we could have a large discrepancy between modal energies $E_i$ and their original potential energies $\Psi_i$. Although the energy regularizer is relatively straightforward, we will show below that some care needs to be taken.

\subsection*{Energy regularization}
Let $\theta = \{ \bm{\Gamma}, \mathbf{W}, \mathbf{b}\}$ be the set of all the parameters of step 2. The effect of the regularizer on this set of parameters with Gradient Descent is written
\begin{equation}
\theta' \leftarrow \theta + \epsilon\lambda_e\nabla_\theta\Omega
\label{eq:update}
\end{equation}
Remember the objective, we want this update to cause lower/higher modal energies $E_i$ for low/high potential energies. To verify this let us compute $\nabla_\theta\Omega$,
\begin{equation}
\nabla_{\theta} \Omega =\sum_{k=1}^M \xi_k \nabla_{\theta} \log(\alpha_k) 
\label{eq:dev}
\end{equation}
The gradient of the logarithm can be developed as
\begin{equation}
\begin{split}  
\nabla_{\theta}  \log(\alpha_k) &= \nabla_{\theta} \log \bigg[ \frac{e^{-\rho E_k}}{Z} \bigg] \\
&=  \nabla_{\theta}(-\rho E_k) -  \nabla_{\theta} \log \sum_{l=1}^M e^{-\rho E_l} \\
&=  -\rho \nabla_{\theta}E_k - \frac{\sum_{l=1}^M \nabla_{\theta} e^{-\rho E_l}}{\sum_{l=1}^M e^{-\rho E_l}} \\
&= -\rho \nabla_{\theta}E_k + \rho \frac{\sum_{l=1}^M e^{-\rho E_l} \nabla_{\theta}E_l}{\sum_{l=1}^M e^{-\rho E_l}} \\
&= \rho \Bigg[ -\big(1 - \frac{e^{-\rho E_k}}{Z}\big)\nabla_{\theta}E_k + \sum_{l \neq k}^M \frac{e^{-\rho E_l}}{Z} \nabla_{\theta}E_l \Bigg] \\
&= \rho \Bigg[ -\big(1 - \alpha_k\big)\nabla_{\theta}E_k + \sum_{l \neq k}^M \alpha_l \nabla_{\theta}E_l \Bigg] \\
\end{split}
\label{eq:grad-log}
\end{equation}
We go further by expressing the equation above with respect to the subset of parameters $\theta_i = \{[\gamma_{ik}, w_{ik}]_{k=1}^M, w_i, b_i\}$:
\begin{equation}
\nabla_{\theta_i}  \log(\alpha_k) = \begin{cases}
      -\rho(1-\alpha_i)\nabla_{\theta_i}E_i, & \text{if}\, i = k \\
       \rho\alpha_i\nabla_{\theta_i}E_i, & \text{if}\, i \neq k
    \end{cases}
\label{eq:log-split}
\end{equation}

The gradient of the regularizer can now be computed by plugging Equation (\ref{eq:log-split}) into the summation (\ref{eq:dev}). Let $M'$ be the number of uncorrupted modes. We obtain for an uncorrupted mode $i$,
\begin{equation}
\nabla_{\theta_i}\Omega = \xi_+\big[ -\rho(1-\alpha_i)\nabla_{\theta_i}E_i \big] + \big[(M'-1)\xi_+ + (M-M')\xi_-\big]\alpha_i\rho\nabla_{\theta_i}E_i
\label{eq:normal-exp}
\end{equation}
and for a corrupted mode $i$,
\begin{equation}
\nabla_{\theta_i}\Omega =\xi_-\big[ -\rho(1-\alpha_i)\nabla_{\theta_i}E_i \big] + \big[M'\xi_+ + (M-M'-1)\xi_-\big]\alpha_i\rho\nabla_{\theta_i}E_i
\label{eq:abnormal-exp}
\end{equation}
Substituting $\xi_k$, we can summarize Equations (\ref{eq:normal-exp}) and (\ref{eq:abnormal-exp}) as
\begin{equation}
\boxed{\nabla_{\theta_i}\Omega = -\big[(M-2M')\alpha_i + \xi_i\big]\rho\nabla_{\theta_i}E_i}
\end{equation}


Adding the constraint that $M' = \lfloor \frac{M+1}{2} \rfloor$, two cases can be distinguished. If the total number of modes $M$ is even, then we have
\begin{equation}
\theta_i' \leftarrow \theta_i - \epsilon\lambda_e\rho\xi_i\nabla_{\theta_i}E_i \quad \text{with} \quad \lambda_e \in \mathbb{R}^+
\end{equation}
Ignoring the second-order effects of the Taylor expansion, we can conclude from the equation above that the regularizer will update the parameters such that modal energies $E_i$ increase/decrease for corrupted/uncorrupted modes $i$.

In analogy, if M is uneven we have
\begin{equation}
\theta_i' \leftarrow \begin{cases}
       \theta_i - \epsilon\lambda_e\rho(1-\alpha_i)\nabla_{\theta_i}E_i, & \text{if $i$ is uncorrupted} \\
       \theta_i + \epsilon\lambda_e\rho(1+\alpha_i)\nabla_{\theta_i}E_i & \text{otherwise}
    \end{cases}
\end{equation}
The principle is the same as in the even case with an additional effect: the correction will be proportional to the error. High energies that have to be low and low energies that have to be high will have stronger gradients than their counterparts. This is similar to the positive and negative phase in the optimization of Restricted Boltzmann Machines.

To conclude, let us notice that some undesired effects can appear if we do not add the constraint $M' = \lfloor \frac{M+1}{2} \rfloor$. As an illustration, take $M' = \lfloor \frac{M+1}{2} \rfloor + 1$, Equation (\ref{eq:update}) becomes
\begin{equation}
\theta_i' \leftarrow \theta_i - \epsilon\lambda_e\rho(\alpha_i + \xi_i)\nabla_{\theta_i}E_i
\end{equation}
which is unstable for uncorrupted modes leading to a collapse where all energies tend to decrease.

%----------------------------------------------------------------------------------------
%	SECTION 
%----------------------------------------------------------------------------------------

\section{Advantages}
Listed below are the advantages of using EMMA instead of standard data-augmentation techniques.
\begin{itemize}
\item The generic design of EMMA permits it to be easily added to any architecture of multi-modal network
\item The burden on the multi-modal network is reduced, it only has to learn to make good predictions from the received information
\item Interpretability is increased, notably the uncertainty on the predictions as we will see in Section \ref{chapter-experiments}
\end{itemize}

%----------------------------------------------------------------------------------------
%	SECTION 
%----------------------------------------------------------------------------------------

\section{Research questions}
\begin{itemize}
\item Does EMMA increase the robustness compared to data augmentation techniques?
\item Is the use of the two regularizers experimentally validated?
\item Is the end-to-end model more interpretable with EMMA?
\end{itemize}

\newpage
\null
\vfill
\begin{center}
\begin{figure}[!h]
\centering
\includegraphics[scale=0.5]{figures/summary-training}
\caption{Summary of end-to-end training}	
\label{fig:training}
\end{figure}
\end{center}
\vfill
\clearpage 
\chapter{Experiments \& Results} 
\label{chapter-experiments} 

Both are experiments on dataset described in previous chapter. Experiment II will be about energy estimation. Experiment III evaluates and analyzes the robustness of the model with and without EMMA.


%----------------------------------------------------------------------------------------
%	SECTION 
%----------------------------------------------------------------------------------------

\section{Pulsar detection}
The models will be trained to detect pulsar stars. In (very) short, pulsar stars are neutron stars emitting radio waves on a periodic time-frame. A summary of the seminal work of \citep{lyon} can be found in Appendix \ref{chapter-dataset}. The thesis can be accessed \href{http://www.scienceguyrob.com/wp-content/uploads/2016/12/WhyArePulsarsHardToFind_Lyon_2016.pdf}{here}\footnote{\url{http://www.scienceguyrob.com/wp-content/uploads/2016/12/WhyArePulsarsHardToFind_Lyon_2016.pdf}} and the dataset \href{https://archive.ics.uci.edu/ml/datasets/HTRU2}{here}\footnote{\url{https://archive.ics.uci.edu/ml/datasets/HTRU2}}.

Short description of two modes (ip and dm) and internal features (mean...).  Explain why detection is difficult. Classification signal/background. Skewed dataset, give numbers.

%----------------------------------------------------------------------------------------
%	SECTION 
%----------------------------------------------------------------------------------------

\section{Corruption}
\begin{itemize}
\item standardize, why? split sets and apply one from train \href{https://stats.stackexchange.com/questions/327294/data-standardization-for-training-and-testing-sets-for-different-scenarios}{error standardize}
\item SNR (see good explanation in pulsar thesis). If greater than 1, signal non-distinguishable. White noise. Explain it is not the same than AE corruption. $ 10\log(\frac{1}{\sigma^2})$
\item on signal and background because we corrupt the whole mode and not the class
\end{itemize}


%----------------------------------------------------------------------------------------
%	SECTION 
%----------------------------------------------------------------------------------------

\section{Experiment II}

\begin{itemize}
\item train-test split
\item AE trained on train-set and then test on test-set
\item matrix with number of signals, ... (eda)
\end{itemize}

Setup: max epochs = 30, batch size = 64, noise DAE = 0.01, d input = 4, n hidden = 12, adam 0.001, sigmoid

\begin{figure}[!h]
\centering
\begin{subfigure}{.5\textwidth}
  \centering
  \includegraphics[width=\linewidth]{figures/noisy-signal-ip}
\end{subfigure}%
\begin{subfigure}{.5\textwidth}
  \centering
  \includegraphics[width=\linewidth]{figures/noisy-signal-dm-snr}
\end{subfigure}
\end{figure}


%----------------------------------------------------------------------------------------
%	SECTION 
%----------------------------------------------------------------------------------------

\section{Experiment III}

\begin{itemize}
\item BCE \& F1 (not AUC -- explain why). \href{https://stats.stackexchange.com/questions/210700/how-to-choose-between-roc-auc-and-f1-score}{F1vsAUC1}. \href{https://www.quora.com/What-does-it-mean-to-have-high-AUC-but-low-F1-score}{F1vsAUC2}. \href{https://stackoverflow.com/questions/44172162/f1-score-vs-roc-auc}{F1vsAUC3}. \href{https://www.mikulskibartosz.name/f1-score-explained/}{F1}
\item train-valid-test
\item threshold optimal choice via ROC. all on valid set
\item 3 models: base (train normal, valid normal), without (train noisy, valid noisy), with (train noisy, valid noisy)
\item 50-25-25 noisy mode. give detailed numbers. eda.
\item trained with early stopping + retrain for .. epochs with valid+train. saved model.
\item one subsection per plot: explain details experiment and how results are obtained. then analyze and conclusions.
\end{itemize}

%\begin{landscape}
%\begin{table}
%\centering
%
%\begin{tabular}{@{}lccccccccc@{}}
%\toprule
%Model &  \multicolumn{3}{c}{$F_1$-Scores} & \multicolumn{3}{c}{Precision} & \multicolumn{3}{c}{Recall}  \\
%\cmidrule(lr){2-4}  \cmidrule(lr){5-7}   \cmidrule(lr){8-10} 
%& uncorrupted & noisy-ip & noisy-dm & uncorrupted & noisy-ip & noisy-dm & uncorrupted & noisy-ip & noisy-dm \\
%\midrule
%Without &  \\ %\cline{1-4}
%\hspace*{\fill}Glasses        & 76.78\% & 70.91\% & 74.17\%& 75.95\% & 76.12\% & 76.04\%& 75.95\% & 76.12\% & 76.04\%  \\ %\cline{1-4}
%With& \\ %\cline{1-4}
%Night-BareFace & 80.66\% & 72.11\% & 77.16\%& 75.95\% & 76.12\% & 76.04\%& 75.95\% & 76.12\% & 76.04\%  \\ %\cline{1-4}
%Night-Glasses  & 74.20\% & 81.66\% & 78.56\%& 75.95\% & 76.12\% & 76.04\% & 75.95\% & 76.12\% & 76.04\% \\ \addlinespace %\cline{1-4}
%Overall        & 75.81\% & 75.65\% & 75.73\%& 75.95\% & 76.12\% & 76.04\%& 75.95\% & 76.12\% & 76.04\% \\ 
%\bottomrule
%\end{tabular}
%\caption{Experimental results}
%\label{my-label}
%\end{table}
%\end{landscape}




  
\subsection*{Attention-shift}
\begin{figure}[!h]
\centering
\begin{subfigure}{.5\textwidth}
  \centering
  \includegraphics[width=\linewidth]{figures/alpha-distrib-high-cap}
\end{subfigure}%
\begin{subfigure}{.5\textwidth}
  \centering
  \includegraphics[width=\linewidth]{figures/beta-distrib-high-cap}
\end{subfigure}
\end{figure}

\begin{figure}[!h]
\centering
\begin{subfigure}{.5\textwidth}
  \centering
  \includegraphics[width=\linewidth]{figures/alpha-distrib-low-cap}
\end{subfigure}%
\begin{subfigure}{.5\textwidth}
  \centering
  \includegraphics[width=\linewidth]{figures/beta-distrib-low-cap}
\end{subfigure}
\end{figure}


\subsection*{Robustness generalisation}
\begin{figure}[!h]
\centering
\begin{subfigure}{.5\textwidth}
  \centering
  \includegraphics[width=\linewidth]{figures/noise-generalisation-bad-model-0}
\end{subfigure}%
\begin{subfigure}{.5\textwidth}
  \centering
  \includegraphics[width=\linewidth]{figures/noise-generalisation-good-model-6}
\end{subfigure}
\end{figure}

\subsection*{Yerkes-Dodson curve}
over-under arousal. do on larger range.
\begin{figure}[!ht]
\centering
\includegraphics[scale=0.5]{figures/yerkes-dodson}
\end{figure}

\subsection*{Energy generalisation}
\begin{figure}[!ht]
\centering
\includegraphics[scale=0.5]{figures/total-energy-model-1}
\end{figure}

 
\chapter{A Unified Model for Multi-Modal Attention} 
\label{chapter-unified} 

The purpose of using EMMA is to help the multi-modal network (MMN) to handle failing modes, but also as a side effect, modes with different contributions and modes with different levels of noise. Chapter \ref{chapter-literature-review} discussed self-attention and crossmodal attention that are used to highlight information inside a specific mode, such as certain regions in an image or a set of frequencies in a sound. The difference between the two is that self-attention uses only the information of the mode itself as a context, whereas crossmodal attention leverages the information in all the modes. We claim to have all the ingredients to construct a complete multi-modal network. As a reminder, human's multi-modal attention consists of three different components: exogenous, endogenous and crossmodal attention. Attention is endogenous when we voluntary choose to attend to something whereas exogenous orients occurs when a person's attention is captured reflexively by the sudden onset of an unexpected event \citep{crossmodal}. Thus, an endogenous module could easily be constructed as a block of $M$ self-attentions. Moreover, EMMA can be considered to be equivalent to exogenous attention. 

\begin{figure}[hbt!]
\centering
\includegraphics[scale=0.5]{figures/unified}
\caption{A possible architecture for a unified multi-modal attention}
\label{fig:complete-model}
\end{figure}

With this in mind, we present a unified model (see Figure \ref{fig:complete-model}) combining all the strengths of each type of attention. First, the attention masks of the exogenous model, $\beta_i$, and the attention masks of the endogenous module, $\mathbf{m}_i$, are combined as $\beta_i\mathbf{m}_i$. The resulting mask is then applied to the input sample, and is passed through the crossmodal module, which explores the relationships between the modes of the input. Finally, the processed input $\{\mathbf{x}'_1\,...\,\mathbf{x}'_M\}$ is forwarded to the MMN. In addition, the complete module can be further refined by inserting feedback loops from the predicted output to the separate modules as it is often done in the literature \citep{afouras, attention-need, bahdanau}. For example, in a self-driving system, the model could adapts its focus to different regions of the input image depending on the previously detected cars. To conclude, let us emphasize that the proposed architecture is only a generalization of attention networds such as \citep{afouras}, supplemented with an exogenous component.


 
\chapter{Conclusion} 
\label{chapter-conclusion} 

The primary objective of this work was to develop a deep learning module whose task is to pre-process multi-modal inputs to reduce the amount of perturbations. In the experiments, it was indeed shown that a masking of the perturbations actually occurs. As a result, the performance of the prediction model on samples with failing modes was improved. Additionally, we experimentally verified that this pre-processing step enables the performance gain to remain stable on more intensive failing modes. In contrast, models trained only with standard data augmentation experienced a decrease of the performance on modes containing more perturbations than in the training set. Despite these promising results, caution must be taken because we were unable to investigate the performance of the module on more complex datasets. Nevertheless, we believe that the ideas detailed in Chapter \ref{chapter-emma} are sufficiently general to be considered as a starting framework for the construction of more robust multi-modal neural networks.

Another main insight was to translate the concept of capacity from psychology to deep learning. This led to the idea of a regularizer forcing the module to limit the amount of extracted information from the input. The concept of capacity and its corresponding regularizer could eventually be applied to deep learning models with a limited amount of processing power as in embedded systems. 

The last contribution proposed an architecture for a unified multi-modal attention, combining the two main types of attention mechanisms found in deep learning (i.e., self and crossmodal) with our attention module.

%----------------------------------------------------------------------------------------
%	SECTION 
%----------------------------------------------------------------------------------------

\section{Future work}
Our results are encouraging but should be validated on more complex datasets containing images, text, sounds, etc. Several points would need to be addressed such as finding efficient methods to approximate the log-likelihood of those data structures\footnote{Several alternatives were proposed in Chapter \ref{chapter-emma}}. Another point is on which level to apply the attention masks, on the raw input data or on the features extracted by the encoders?

An idea that came up during this thesis but was not tested, is based on the following observation: the transition between the first and second stage of the training can be quiet "brutal" for the MMN. Indeed, weights of the MMN at the start of the second stage are optimal for uncorrupted modes. However, these weights will have to adapt immediately to weighted inputs, as an effect of EMMA. A smoother approach to consider is inspired from the process of \textit{annealing} in metallurgy; "Annealing is a process in which a solid is first heated until all particles are randomly arranged in a liquid state, followed by a slow cooling process. At each cooling temperature enough time is spend for the solid to reach thermal equilibrium."\footnote{Explanation from \href{http://www.iue.tuwien.ac.at/phd/binder/node87.html}{here}} Applying this idea to our case, we could divide the weights of the first layer of the MMN by $\tanh(1/M)$ at the start of the second stage, and set the temperature high enough to obtain a uniform distribution of importance scores ($\alpha_i \approx 1/M,\, \forall i$). As a consequence, the effect of EMMA on the inputs of the MMN would be non-existent\footnote{This assumption is only guaranteed if the parameters $g_a$ and $b_a$ are intialized as $g_a=1$ and $b_a=0$.}, keeping the latter in its local minima. Subsequently, a cooldown schedule would be applied to the temperature leading smoothly to more pronounced attention shifts on the input data. This guided approach has no guarantee to improve the results but in our opinion can be worth trying. Moreover, instead of fixing a final temperature\footnote{Temperature at which the cooling schedule is stopped.} by tuning, a similar approach to Early Stopping could be used: the cooldown would be stopped when the validation error stops decreasing significantly.

 

%----------------------------------------------------------------------------------------
%	THESIS CONTENT - APPENDICES
%----------------------------------------------------------------------------------------

\appendix % Cue to tell LaTeX that the following "chapters" are Appendices

% Include the appendices of the thesis as separate files from the Appendices folder
% Uncomment the lines as you write the Appendices

\include{appendices/appendixA}
\include{appendices/appendixB}
\include{appendices/appendixC}

%----------------------------------------------------------------------------------------
%	BIBLIOGRAPHY
%----------------------------------------------------------------------------------------

\printbibliography[heading=bibintoc]

%----------------------------------------------------------------------------------------

\end{document}  
