\chapter{Conclusion} 
\label{chapter-conclusion} 

Summary of what was seen/done during the master thesis from start to end. Scales up quadratically with number of modes, explain.

%----------------------------------------------------------------------------------------
%	SECTION 
%----------------------------------------------------------------------------------------

\section{Contributions}
Summarize contributions


%----------------------------------------------------------------------------------------
%	SECTION 
%----------------------------------------------------------------------------------------

\section{Research questions}
Each research question + answer


%----------------------------------------------------------------------------------------
%	SECTION 
%----------------------------------------------------------------------------------------

\section{Future work}
\begin{itemize}
\item Annealing + init, end temperature + explain init of next layer \href{http://what-when-how.com/artificial-intelligence/a-comparison-of-cooling-schedules-for-simulated-annealing-artificial-intelligence/}{Linke annealing}. Multiple modes then blows up. Image 100 modes (altough not realistic in real-world problems) then tend to zero. Solution add a common gain? Analyze influence of multiple modes etc..

\item Explore different shared energies design

\item Images/sound sequences. early, late fusion, manifold with respect to unified model? it could be very easy to test on images/sounds (give even during test-time a true outlyingness measure, not specifically the NLL) and at the same time investigate general ways of getting such a measure.

\item Unseen values
\end{itemize}


